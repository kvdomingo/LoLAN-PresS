\documentclass[12pt,a4paper]{article}
% The following LaTeX packages must be installed on your machine: amsmath, authblk, bm, booktabs, caption, dcolumn, fancyhdr, geometry, graphicx, hyperref, latexsym, natbib
\input{185.dat}
\usepackage{gensymb}
\usepackage{amsthm}
\usepackage{float}
\usepackage{siunitx}
\usepackage{amssymb}
\usepackage{float}
\usepackage{enumerate}
\usepackage{listings}
\usepackage{mathtools}
\PassOptionsToPackage{hyphens}{url}\usepackage{hyperref}
\usepackage[none]{hyphenat}
\usepackage{physics}
\newcommand\ddfrac[2]{\frac{\displaystyle #1}{\displaystyle #2}}
%\renewcommand{\familydefault}{\sfdefault}


\begin{document}

\begin{titlepage}
\begin{center}
\vspace*{\fill}

\Huge{ App Physics 185 \\
Capstone Project
} \\

\qquad
\qquad

\normalsize{Kenneth V. Domingo \\
Rhei Joven G. Juan \\
Rene L. Principe Jr.
}

\vspace*{\fill}
\end{center}
\end{titlepage}

\setcounter{page}{1}

\section*{Week of 6 May 2019: Update 1}
\medskip

\subsection*{Materials}
\begin{table}[h!]
	\centering
	\caption{Items required and expenses so far.}
	\begin{tabular}{cccc}
	Quantity & Item & Cost/pc (PhP) & Subtotal (PhP) \\ \hline
	1 & Velostat sheet & 349.00 & 349.00 \\
	1 & Heltec ESP32 LoRa Microcontroller & 0.00$^*$ & 0.00 \\
	1 & Aluminum tape & 0.00$^{**}$ & 0.00 \\
	& & \textbf{TOTAL} & 349.00 \\
	& $_{^*\textit{provided by Dr. Tapang}}$ & & \\
	& $_{^{**}\textit{stock available in lab}}$ & &
	\end{tabular}
\end{table}

\subsection*{Tasks accomplished}
\begin{itemize}
	\item Familiarized ourselves with how to construct a sensor based on velostat pressure sheets
	\item Refreshed our knowledge of programming in C++ (for Arduino)
	\item Downloaded and installed drivers specific to the microcontroller and familiarized ourselves with its pinout
	\item Were able to construct a basic velostat sensor and display its analog input reading both on the serial monitor and the microcontroller LED
\end{itemize}

\subsection*{Results}
\begin{itemize}
	\item Analog input of microcontroller $\in [0, 4095]$ \texttt{int}
	\item Readings were scaled such that the range $\in [0, 5]$ V \texttt{float}
	\item Maximum reading is obtained when there is no applied pressure, reading decreases as pressure is increased
	\item Continuous serial plot shows a constant value when no pressure is applied, but changing the pressure shows that the analog reading is not continuous when pressure is applied: the reading constantly jumps between a certain value corresponding to the pressure and the maximum value, i.e. applying a constant force on the sheet at rest does not produce an expected step function, but rather, uneven spikes
\end{itemize}

\subsection*{Up next}
\begin{itemize}
	\item Find a way to stabilize pressure readings to get a proper constant reading when applying constant pressure
	\item Develop code for recognition of type of motion based on the appearance of voltage vs time graph by evaluating it for every time period $\tau$
	\item Attempt to calculate time of flight based on voltage vs time graph
\end{itemize}

\lstset{
	basicstyle=\footnotesize\ttfamily,
	language=C++,
	tabsize=4,
	numbers=left,
	numberstyle=\tiny\ttfamily,
	caption={Inital code for familiarization of microcontroller.},
	label=list:source
}
\begin{lstlisting}
#include <heltec.h>

float Vin = 0;
float Vscaled = 0;
int READPIN = 0;

void setup() {
	Serial.begin(9600);
	Heltec.begin(true, false, true);
	Heltec.display->flipScreenVertically();
	Heltec.display->setFont(ArialMT_Plain_16);
	pinMode(READPIN, INPUT);
}

void loop() {
	Heltec.display->clear();
	Vin = analogRead(READPIN);
	Vscaled = Vin/4095. * 5.;
	Serial.println(Vin);
	Heltec.display->drawString(0, 10, String(Vscaled));
	Heltec.display->display();
}
\end{lstlisting}


\clearpage

\section*{Week of 29 April 2019: Proposal}
\medskip

\subsection*{Objectives}
\begin{itemize}
	\item To be able to detect the force exerted by various locations on a person's foot.
	\item To be able to characterize, based on the force vs time graph, whether a person is standing, walking, running, or jumping.
	\item To be able to characterize the sport being played based on the force vs time graph.
\end{itemize}

\subsection*{Materials \& Methodology}

\begin{table}[h!]
	\centering
	\caption{Estimated cost of materials.}
	\begin{tabular}{cccc}
	Quantity & Item & Cost/pc (PhP) & Subtotal (PhP) \\ \hline
	1 & Pressure-sensitive conductive velostat sheet & 349.00 & 349.00 \\
	1 & ESP8266 WiFi Microcontroller & 325.00 & 325.00 \\
	2 & 9V Battery & 79.00 & 158.00 \\
	& & \textbf{TOTAL} & 832.00
	\end{tabular}
\end{table}

\begin{enumerate}
	\item Pieces of the velostat sheet are attached to 3 locations on the insole of a shoe. 
	\item The velostat is attached to the microcontroller for real-time data acquisition.
	\item Voltage vs. force calibration was done by applying different values of force to the velostat sheet.
	\item A force vs. time is plotted in real-time for different motions.
	\item A characterization of the different motions was done using the frequencies and magnitudes obtained from the force vs. time plot.
\end{enumerate}

\subsection*{Predicted Results}
Characterization of different motions:
\begin{itemize}
	\item \textbf{Standing} - constant force observed over time
	\item \textbf{Walking} - a cascading motion from the three sensors will be observed
	\item \textbf{Running} - same as walking but with higher frequency
	\item \textbf{Jumping} - two sets of impulse will be observed from the take-off and landing.
\end{itemize}

\subsection*{Validation Scheme}
\begin{itemize}
	\item Compare the obtained force value from the sensor with the calculated calibration equation/curve, for standing motion.
	\item Through real-time analysis, compare a video of the motions to their force vs. time plot.
\end{itemize}

\end{document}